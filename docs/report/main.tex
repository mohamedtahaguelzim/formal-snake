\documentclass[letterpaper, 11pt]{article}
\usepackage{style}
\usepackage{tgpagella}
% \setlength{\parindent}{0pt}

\title{Formal Verification of the Snake Game as a State Machine}
\author{
  Clément Yves Pierre Tsedri \\
      \texttt{\href{mailto:clement.tsedri@epfl.ch}{clement.tsedri@epfl.ch}} \and
  Mohamed Taha Guelzim \\
      \texttt{\href{mailto:mohamed.guelzim@epfl.ch}{mohamed.guelzim@epfl.ch}} \and
  Ozair Faizan \\
      \texttt{\href{mailto:ahmad.faizan@epfl.ch}{ahmad.faizan@epfl.ch}}
}
\date{}

\begin{document}
\maketitle
\setlength{\belowdisplayskip}{4pt} \setlength{\belowdisplayshortskip}{4pt}
\setlength{\abovedisplayskip}{4pt} \setlength{\abovedisplayshortskip}{4pt}


\abstract{
    The Snake game, while considered trivial software, present complex environment and state spaces
    which makes it an excellent candidate for formal verification case study.
    In this project, we formalize the Snake game as a state machine and verify the correctness of
    its game logic using Stainless, a verification framework for Scala.
    We define a set of invariants describing a valid playging game state and prove that the game's
    transition function preserves these invariants along with correct transitions to terminal states
    (Game Over or Game Won).
}


\section{Introduction}
Formal verification is often associated with safety-critical systems, however, applying these
methods to game development offers a unique pedagogical opportunity to study state-based reasoning
and invariant preservation in a system with immediate visual feedback and intuitive rules. 

The Snake game involves a player-controlled snake moving on a grid, growing as it consumes food, and
dying if it collides with walls or itself. While traditionally a real-time game, our implementation
focuses on a turn-based version, meaning the game state updates only when the player provides input
(although the game is playable in real-time too).

Despite its apparent simplicity, the state space is vast, making exhaustive testing impossible.
In this project, we aim to verify that our implementation of Snake never enters an undefined or
illegal state during execution. We model the game in Scala using immutable data structures and use
the Stainless framework to prove inductive invariants (safety properties).


\section{Model and Game Logic}
The game is modeled as a finite state machine, the state is captured by
\begin{minted}{scala}
    case class GameState(
        snake: List[Position],
        food: Option[Position],
        direction: Direction,
        status: GameStatus,
        config: GameConfig,
        stateNumber: BigInt,
        pendingDirection: Option[Direction]
    )
\end{minted}
where \texttt{status} tracks the high-level state described in Figure~\ref{fig:status}.
The core verification focuses on the \mintinline{scala}{Playing} status, where the logic is
non-trivial.

The function \mintinline{scala}{def processGameTick(state: GameState, foodSeed: BigInt): GameState}
advances the game by one tick while in a \mintinline{scala}{Playing} status.
It computes the next head position, checks for collisions, moves the snake, handles wins, and may
generate new food. The snake movement is handled as follows, if food was eaten then the new snake is
\mintinline{scala}{nextHead :: oldSnake} otherwise it is \mintinline{scala}{nextHead ::
withoutLast(oldSnake)} where \texttt{withoutLast} removes the last element of the list.

The function \mintinline{scala}{def generateFood(state: GameState, seed: BigInt): Position} outputs
a grid position where the food will be place such that it is not occupied by the snake. It uses
\texttt{seed} which is a pseudo-random number.

The verification of the above two function is the core of our project.

\begin{figure}[ht]
    \centering
    \begin{tikzpicture}[
        ->,
        >=stealth,
        node distance=2.8cm,
        state/.style={circle, draw, minimum size=1.2cm, font=\small},
        initial/.style={state, fill=gray!20}
    ]

    \node[initial] (waiting) {Waiting};
    \node[state] (playing) [right=of waiting] {Playing};
    \node[state] (gameover) [below right=1.5cm and 1cm of playing] {GameOver};
    \node[state] (gamewon) [above right=1.5cm and 1cm of playing] {GameWon};

    \path
        (waiting) edge node[above, font=\tiny] {startGame} (playing)
        (playing) edge[loop above] node[above, font=\tiny] {tick} (playing)
        (playing) edge node[below left, font=\tiny] {collision} (gameover)
        (playing) edge node[above left, font=\tiny] {win} (gamewon)
        (gameover) edge[bend left=35] node[below, font=\tiny] {reset} (waiting)
        (gamewon) edge[bend right=35] node[above, font=\tiny] {reset} (waiting);

    \node[draw, rectangle, dashed, align=left, font=\tiny, right=0.2cm of playing] {
      \textbf{validPlayingState}
      };
    \end{tikzpicture}
    \caption{High-level state diagram of the game.}
    \label{fig:status}
\end{figure}


\section{Specifications and Invariants}
We require that every state in \mintinline{scala}{Playing} status satisfies
\begin{minted}{scala}
  def validPlayingState(state: GameState): Boolean =
    state.status == GameStatus.Playing &&
      0 < state.snake.length &&
      state.snake.length < state.gridWidth * state.gridHeight &&
      allWithinBounds(state.snake, state.gridWidth, state.gridHeight) &&
      noSelfIntersection(state.snake) &&
      contiguous(state.snake)
\end{minted}
which serves as the central invariant.
It states that the snake length is strictly positive and less that the total grid size $(W \times
H)$, all snake positions are within the grid boundaries, the snake has no duplicate positions and
each segment of the snake is adjacent to the next segment.

Furthermore, we require \mintinline{scala}{processGameTick} to satisfies
\mintinline{scala}{validTransistion} which describes valid transitions, that is,
\begin{itemize}
    \item The snake length never decreases and grows by at most one
    \item Reaching \mintinline{scala}{GameOver} is equivalent to \texttt{nextHead} colliding with
        the walls or with the snake
    \item Reaching \mintinline{scala}{GameWon} is equivalent to the snake filling the grid
    \item The tail of the new snake must be a subsequence of the previous snake
        (movement preserves body).
\end{itemize}

For a \mintinline{scala}{Playing} state, the \mintinline{scala}{generateFood} function can always
return a valid position to place the food since the length of the snake is strictly less than the
size of the grid, however, as we will see, the proof of this statement is non-trivial.


\section{Verification of Game Logic}
\subsection{Initial Playing State is Valid}

\subsection{Preservation of Invariants}

\subsection{Transition Safety}

\subsection{Food Generation Safety}


\section{Results and Metrics}


\section{Conclusion}


\end{document}
