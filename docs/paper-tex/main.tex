\documentclass[letterpaper, 11pt]{article}
\usepackage{style}
\usepackage{tgpagella}
% \setlength{\parindent}{0pt}

\title{Formal Verification of the Snake Game as a State Machine}
\author{
  Clément Yves Pierre Tsedri \\
      \texttt{\href{mailto:clement.tsedri@epfl.ch}{clement.tsedri@epfl.ch}} \and
  Mohamed Taha Guelzim \\
      \texttt{\href{mailto:mohamed.guelzim@epfl.ch}{mohamed.guelzim@epfl.ch}} \and
  Ozair Faizan \\
      \texttt{\href{mailto:ahmad.faizan@epfl.ch}{ahmad.faizan@epfl.ch}}
}
\date{}

\begin{document}
\maketitle
\setlength{\belowdisplayskip}{4pt} \setlength{\belowdisplayshortskip}{4pt}
\setlength{\abovedisplayskip}{4pt} \setlength{\abovedisplayshortskip}{4pt}

\abstract{
    The Snake game offers a simple yet non-trivial environment for studying state-based reasoning, safety properties, and termination. In this project, we intend to formalize Snake as a State Machine and verify the correctness of its controller and high-level properties using Stainless, a verification framework for Scala. The set of possible states will have one invalid state and valid states that encode the snake, the walls and the apples on the grid. \S\ref{sec:prop} gives some interesting properties we would like to prove.

    As a background reference, we will review the paper \cite{games-using-fm}, which formalizes simple games such as Pac-Man and Chess in ProB.
}


\section{Properties we will verify}\label{sec:prop}
We model the snake as \mintinline{scala}{List[(Position, Direction)]} where the list is ordered from
the head of the snake to its tail.
\begin{itemize}
    \item The snake is continuous, that is, the $i$-th position of the snake is given by the
        $i+1$-th position plus the $i+1$-th direction for $0 \leqslant i < \text{snake length} - 1$ 
    \item The snake does not collide with itself or the walls, and remains inside the grid.
    \item The transition function of the state machine outputs the invalid state from a valid state
        only if one of the above conditions is not satisfied for the updated state.
    \item Valid moves are defined according to \textbf{Definition 1} in \cite{155769e4c59d41deb4982a73745c5b88}.
    \item Additional safety and liveness properties may be considered as the model evolves.
\end{itemize}



\printbibliography
\end{document}
